\chapter*{ВВЕДЕНИЕ}
\addcontentsline{toc}{chapter}{ВВЕДЕНИЕ}

Шифрование информации --- занятие, которым человек занимался ещё до начала первого тысячелетия, занятие, позволяющее защитить информацию от посторонних лиц. 

Криптографический алгоритм RSA --- алгоритм, разработанный в 1977 году и положивший основу первой системе, пригодной как для шифрованя, так и для цифрвой подпис

Хеширование --- процесс преобразования набора данных произвольной длины в выходной набор данных установленной длины, выполняемый определённым алгоритмом.

\textbf{Целью данной работы} является реализация в виде программы на языке программирования C или C++, позволяющую создать и проверить электронную подпись для документа с использованием алгоритма RSA и алгоритма хеширования SHA1 или MD5.

Для достижения поставленной цели необходимо выполнить следующие задачи:
\begin{enumerate}[label=\arabic*)]
	\item изучить криптографический алгоритм RSA и алгоритм хеширования SHA1 или MD5;
	\item реализовать криптографический алгоритм RSA в виде программы, обеспечив возможности создания и проверки подлинности электронной подписи для документа с использованием алгоритма SHA1 или MD5;
	\item протестировать разработанную программу, показать, что удаётся создавать и проверять электронные подписи;
	\item описать и обосновать полученные результаты в отчёте о выполненной лабораторной работе.
\end{enumerate}
