\chapter{Технологическая часть}

\section{Средства реализации}

Для программной реализации шифровальной машины был выбран язык C++ \cite{cpp}.
В данном языке есть все требующиеся инструменты для данной лабораторной работы.
В качестве среды разработки была выбрана среда CLion \cite{clion}.

\section{Реализация алгоритма}

\begin{lstlisting}[label=lst:enigma:reflactor,caption=Класс рефлектора]
class Reflector {
public:
	explicit Reflector();
	explicit Reflector(const std::vector<uint8_t>& config);
	explicit Reflector(const std::string& config);
	
	uint8_t reflect(uint8_t symbol);
	void printf_config();
private:
	std::vector<uint8_t> _config;
};

Reflector::Reflector(const std::vector<uint8_t>& config): _config(config) {}

uint8_t Reflector::reflect(uint8_t symbol) {
	return _config[symbol];
}
\end{lstlisting}

\clearpage

\begin{lstlisting}[label=lst:enigma:rotor,caption=Класс ротора]
class Rotor {
public:
	explicit Rotor(const std::vector<uint8_t>& wiring);
	
	uint8_t encrypt_left(uint8_t symbol);
	uint8_t encrypt_right(uint8_t symbol);
	
	void set_position(uint8_t position);
	void reset_position();
	void rotate();
private:
	uint8_t find_index(uint8_t letter);
	std::vector<uint8_t> _wiring;
	std::vector<uint8_t> _start;
	size_t _size;
};


Rotor::Rotor(const std::vector<uint8_t>& wiring): _wiring(wiring), _start(_wiring) {	
	this->_size = this->_wiring.size();
}

uint8_t Rotor::encrypt_right(uint8_t symbol) {
	return _wiring[symbol];
}

uint8_t Rotor::encrypt_left(uint8_t symbol) {
	return find_index(symbol);
}

uint8_t Rotor::find_index(uint8_t letter) {
	for (int i = 0; i < _size; i++)
	{
		if (_wiring[i] == letter)
		{
			return i;
		}
	}
		
	return -1;
}

void Rotor::reset_position() {
	_wiring = _start;
}

void Rotor::rotate()
{
	uint8_t temp = _wiring[_size - 1];
	for (int i = _size - 1; i > 0; --i) {
		_wiring[i] = _wiring[i - 1];
	}
	_wiring[0] = temp;
}
\end{lstlisting}

\begin{lstlisting}[label=lst:enigma:reflactor,caption=Класс енигмы]
class Enigma {
public:
	Enigma();
	Enigma(uint64_t size_rotor, uint8_t amount_rotors);
	
	void set_reflector(Reflector& reflector);
	void set_commutator(Reflector& reflector);
	void set_rotor(Rotor& rotor);
	
	size_t encrypt(FILE *fin, FILE *fout);
	std::string encrypt(const std::string& message);
	uint8_t encrypt(uint8_t symbol);
	void reset_rotors();
	
void printf_config();
	private:
	void print_pair(uint8_t s1, uint8_t s2);
	char normalize_sym(uint8_t symbol);
	int _counter;
	int _size_rotor;
	uint8_t _amount_rotors;
	std::unique_ptr<Reflector> _reflector;
	std::unique_ptr<Reflector> _commutator;
	std::vector<std::shared_ptr<Rotor>> _rotors;
};

Enigma::Enigma(uint64_t size_rotor, uint8_t amount_rotors) {
	this->_counter = 0;
	this->_size_rotor = (int) size_rotor;
	this->_amount_rotors = amount_rotors;
}

void Enigma::set_reflector(Reflector &reflector) {
	_reflector = std::make_unique<Reflector>(reflector);
}

void Enigma::set_commutator(Reflector &reflector) {
	_commutator = std::make_unique<Reflector>(reflector);
}

void Enigma::set_rotor(Rotor &rotor) {
	_rotors.push_back(std::make_shared<Rotor>(rotor));
}

uint8_t Enigma::encrypt(uint8_t symbol)
{
	int rotor_queue = 0;
	uint8_t new_symbol = symbol;
	
	new_symbol = _commutator->reflect(new_symbol);
	
	for (int i = 0; i < _amount_rotors; i++)
	{
		new_symbol = _rotors[i]->encrypt_left(new_symbol);
	}
	
	new_symbol = _reflector->reflect(new_symbol);
	

	for (int i = _amount_rotors - 1; i >= 0; i--)
	{
		new_symbol = _rotors[i]->encrypt_right(new_symbol);
	}
	
	new_symbol = _commutator->reflect(new_symbol);
	
	rotor_queue = 1;
	this->_counter += 1;
	for (int i = 0; i < _amount_rotors; ++i) {
		if (_counter % rotor_queue == 0) {
			_rotors[i]->rotate();
		}
		rotor_queue *= _size_rotor;
	}
	
	return new_symbol;
}

void Enigma::reset_rotors() {
	for (auto rotor: _rotors)
	{
		rotor->reset_position();
	}
}

std::string Enigma::encrypt(const std::string& message) {
	std::string new_message;
	for (char symbol: message)
	{
		new_message += static_cast<char>(encrypt(symbol));;
	}
	
	return new_message;
}

size_t Enigma::encrypt(FILE *fin, FILE *fout) {
	if (fin == nullptr || fout == nullptr)
	{
		return -1;
	}
	
	std::wstring message;
	
	char code;
	fseek(fin , 0, SEEK_END);
	size_t input_size = ftell(fin);
	fseek(fin , 0, SEEK_SET);
	size_t size = 0;

	
	while (size < input_size) {
		size += fread(&code, sizeof(char), 1, fin);
		fseek(fin , SEEK_SET, SEEK_CUR);
		
		uint8_t newcode = this->encrypt(code);
		
		fwrite(&newcode, sizeof(char), 1, fout);
	}
	
	return size;
}
\end{lstlisting}

\section{Тестирование}

\begin{table}[ht!]
	\begin{center}
		\captionsetup{justification=raggedright,singlelinecheck=off}
		\caption{\label{tbl:functional_test} Функциональные тесты}
		\begin{tabular}{|c|c|c|}
			\hline
			Входная строка & Выходная строка \\ 
			\hline
			HeLlo WorLd & >>кйД;пољ6 \\
			ABOBA  & ґµ;№M\\
			ґµ;№M  & ABOBA \\
			A & ґ\\
			<<>>  & <<>>\\
			\hline
		\end{tabular}
	\end{center}
\end{table}