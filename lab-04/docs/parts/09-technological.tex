\chapter{Технологическая часть}

\section{Средства реализации}

Для программной реализации шифровальной машины был выбран язык C++ \cite{cpp}.
В данном языке есть все требующиеся инструменты для данной лабораторной работы.
В качестве среды разработки была выбрана среда CLion \cite{clion}.

\section{Реализация алгоритма}

\begin{lstlisting}[label=lst:aes,caption=Генерация ключей RSA]
std::pair<PrivateKey, PublicKey> RSA::genPublicAndSecretKeys(int64_t min_value, int64_t max_value) {
	auto [p, q] = genPQSimple(min_value, max_value);
	
	int64_t N = p * q;
	
	int64_t euler = (p - 1) * (q - 1);
	
	std::mt19937 generator(std::random_device{}());
	std::uniform_int_distribution<std::int64_t> distribution(2, euler - 1);
	
	int64_t e = distribution(generator);
	while (gcd(e, euler) != 1) {
		e = distribution(generator);
	}
	
	int64_t d = std::get<1>(extendet_efclid_alg(e, euler));
	if (d < 0) {
		d += euler;
	}
	
	return {
		PrivateKey{
			d, N
		},
		PublicKey{
			e, N
		}
	};
}
\end{lstlisting}