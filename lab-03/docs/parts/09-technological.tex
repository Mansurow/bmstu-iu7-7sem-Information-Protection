\chapter{Технологическая часть}

\section{Средства реализации}

Для программной реализации шифровальной машины был выбран язык C++ \cite{cpp}.
В данном языке есть все требующиеся инструменты для данной лабораторной работы.
В качестве среды разработки была выбрана среда CLion \cite{clion}.

\section{Реализация алгоритма}

\begin{lstlisting}[label=lst:CFB,caption=Класс реализации режима CFB]
class cfb {
	public:
	explicit cfb (const vector<uint8_t>& vi);
	
	string crypt(string message, bool decrypt = false);
	
	vector<block> divideBlocks(const vector<uint8_t> &message);
	vector<uint8_t> mergeBlocks(const vector<block> &blocks);
	
	vector<uint8_t> encrypt(vector<uint8_t>& block128, const vector<uint8_t>& key);
	vector<uint8_t> decrypt(vector<uint8_t>& block128, const vector<uint8_t>& key);
	
	void setVI(const vector<uint8_t>& vi);
	
	void print_bloks(const vector<block> &blocks);
	void print_message(const vector<uint8_t> message);
};
\end{lstlisting}


\clearpage

\begin{lstlisting}[label=lst:aes,caption=Класс шифрования и дешифрования AES]
class aes: public IEncoder{
	public:
	block CryptBlock(const block& block128, const key& key128, bool decrypter = false);
	block EncryptBlock(const block& block128, const key& key128) override;
	block DecryptBlock(const block& block128, const key& key128) override;
	
	vector<key> GetKeys128(key key128, bool decrypter = false);
	void AddRoundKey(mtx& block, const key& roundKey);
	void SubBytes(mtx& block);
	void InvSubBytes(mtx& block);
	void ShiftRows(mtx& block);
	void InvShiftRows(mtx& block);
	void MixColumns(mtx& block);
	void InvMixColumns(mtx& block);
	
	mtx ArrayToMrx4x4(const block& block);
	block Mrx4x4ToArray(const mtx &mtx);
	void AddPadding(vector<uint8_t>& data);
	void RemovePadding(vector<uint8_t>& data);
	private:
	// Уожение поля Галуа
	uint8_t GMul(uint8_t x, uint8_t y);
	
	AES_MODE mode = AES128;
}
\end{lstlisting}

\clearpage

\begin{lstlisting}[label=lst:AES-m,caption=Реализация алгоритма AES]
block aes::EncryptBlock(const block &block128, const key &key128) {
	// Расширение ключа - KeyExpansion
	auto keys = GetKeys128(key128, false);
	
	// для удобства конвертируем в матрицу
	auto state = ArrayToMrx4x4(block128);
	
	// Начальный раунд - сложение с основным ключом;
	AddRoundKey(state, keys[0]);
	
	// 9 раундов шифрования
	for (int i = 1; i < 10; i++)
	{
		SubBytes(state);
		ShiftRows(state);
		MixColumns(state);
		AddRoundKey(state, keys[i]);
	}
	
	// Финальный раунд
	SubBytes(state);
	ShiftRows(state);
	AddRoundKey(state, keys[10]);
	
	return Mrx4x4ToArray(state);
}
\end{lstlisting}


\section{Тестирование}

Тестирование разработанной программы производилось следующим образом: выбирались случайные значения ключа и вектора IV, а также получалась случайная последовательность блоков для шифрования длиной $n$. Она зашифровывалась и расшифровывалась, проверялось совпадение полученного результата с начальными данными. Данная процедура повторялась $n$ раз для значений $n$ от 1 до 100.

\begin{table}[h]
	\begin{center}
		\begin{threeparttable}
			\captionsetup{justification=raggedright,singlelinecheck=off}
			\caption{\label{tbl:functional_test} Функциональные тесты}
			\begin{tabular}{|c@{\hspace{7mm}}|c@{\hspace{7mm}}|c@{\hspace{7mm}}|c@{\hspace{7mm}}|c@{\hspace{7mm}}|c@{\hspace{7mm}}|}
				\hline
				Длина, байты & Шифруемое значение & Результат работы \\ 
				\hline
				
				8 &
				12345678 &
				ҐЯЄ}70ЌЗ \\ \hline
				16 & 1234567812345678 &
				\begin{tabular}{c}
					ҐЯЄ}70ЌЗРx~¬6@‚\\
				\end{tabular}   \\ \hline
				
				32 &
				\begin{tabular}{c}
					1234567812345678\\
					1234567812345678
				\end{tabular} &
				\begin{tabular}{c}
					ҐЯЄ}70ЌЗРx~¬6@‚\\
					vќ F°qћЎѕђGZд/\$
				\end{tabular} \\ \hline
				
			\end{tabular}
		\end{threeparttable}
	\end{center}
	
\end{table}


\section*{Вывод}

В данном разделе были рассмотрены средства реализации, а также представлены листинги реализации шифровального алгоритма AES и режима работы CFB,  произведено тестирование.