\chapter{Аналитическая часть}

В этом разделе будут рассмотрены классический алгоритм работы шифровальной машины <<Энигма>>, а также её вариант, использованный во время Второй мировой войны, приведён пример преобразования буквы, а также подсчитано количество комбинаций <<Энигмы>> с 3 роторами.


\section{Механизмы шифрования}

Шифровальная машина «Энигма» внешне выглядит как печатающая машинка, за исключением того факта, что шифруемые символы не печатаются автоматически на определённый лист бумаги, а указываются на панели посредством загорания лампочки.

Шифровальная машина «Энигма» обладает тремя основными механизмами.
\begin{enumerate}
	\item Роторы --- cердце всех шифровальных машин, которое со стороны классической криптографии они реализуют полиалфавитный алгоритм шифрования, а их определённо выстроенная позиция представляет собой один из основных ключей шифрования. 
	Каждый ротор не эквивалентен другому ротору, потому как обладает своей специфичной настройкой. Военным на выбор давалось пять роторов, три из которых они вставляли в «Энигму».
	\item Рефлектор --- cтатичный механизм, позволяющий шифровальным машинам типа «Энигма» не вводить помимо операции шифрования дополнительную операцию расшифрования. Связано это с тем, что в терминологии классической криптографии рефлектор представляет собой просто частный случай моноалфавитного шифра.
	\item Коммутатор позволяет оператору шифровальной машины варьировать содержимое проводов, попарно соединяющих буквы английского алфавита.
\end{enumerate}

\section{Алгоритм работы шифрования}

В данной работе будет подразумеваться, что у оператора машины состоит из 3 роторов и 1 рефлекторов, а также 26 соединительных проводов для коммутационной панели:
\begin{itemize}
	\item на вход поступает файл с данными и посимвольно считывается;
	\item каждый символ поступает в коммутационную панель, благодаря чему поставляется постановленый парный код символа;
	\item затем данный код поступает в каждый ротор, где осуществляется преобразованые в новый код символа;
	\item после 3 роторов код символа поступает в рефлектор и сопоставляется парный код символа;
	\item данный код в обратном направлении проходит через все роторы;
	\item новый код поступает в коммутатор и ему сопоставляется соотвествующая пара;
	\item получаем шифрованную букву;
	\item первый ротор поворачивается на одну позицию, если один ротор совершит полный оборот всех позиций, то менять позицию начнет следующий ротор. 
\end{itemize}